\documentclass{article}
\usepackage{amsmath, amsthm, amssymb, amsfonts, dsfont, fancyhdr, graphicx, color, tabularx, enumitem}
\usepackage{geometry}


\theoremstyle{definition}
\newtheorem{prob}{}
\renewcommand{\qedsymbol}{}
\renewcommand*{\proofname}{Solution}
\newcommand{\MSB}[1]{\textcolor{blue}{[MSB: #1]}}

\pagestyle{fancy} \fancyhf{} \lhead{\textsc{18.06}} \rhead{9/13/22} 

\begin{document}


\section*{Practice Problems}
\begin{prob}[Lecture recap---skip if you feel like it] A function $f$ is called ``linear" if $f(x+y)=f(x) + f(y)$ and $f(cx)=c f(x)$ for any scalar $c$.
	\begin{itemize}
		\item [a)] Is $f(x)=m x +b$ linear? What about $f(x)=x^2$? (In both cases, $f$ is a function on the real numbers)
		\item [b)] Show that $f(x)=Ax$ is linear for any $2 \times 2$ matrix $A$. (Here, $x$ is any $2 \times 1$ vector.)
		\item [c)] Show that $f(X)=AX$ is linear for any $2 \times 2$ matrix $A$. (Here, $X$ is any $2 \times 2$ matrix.)
	\end{itemize}
\end{prob}

\begin{prob}
	Say $x, y, z$ are $4$-component column vectors. The equation
	$$x(y+z)=xy + xz = yx + zx$$
	is nonsense (why?) but is a few symbols away from being true. Decorate with transposes to make it a true equation.
\end{prob}
 

\begin{prob}
Say $P$ is the $4 \times 4$ linear operation that reverses the order, i.e.
\[P \begin{bmatrix}
	x_1 \\ x_2 \\ x_3\\ x_4
\end{bmatrix}= \begin{bmatrix}
x_4 \\ x_3 \\ x_2\\ x_1
\end{bmatrix}.\]
What does $P$ do to the $4 \times 4$ identity matrix $I$? How can you use this to write down $P$?

More generally, if you know how a linear operation $A$ behaves on a vector of variables, how can you write down the matrix for $A$?
\end{prob}

\begin{prob}
	Find the $LU$ factorization of 
	\[A=\begin{bmatrix}
		a & a & a\\
		a & b & b\\
		a & b& c
	\end{bmatrix}.\]
What 3 conditions on $a, b, c$ guarantee that $A=LU$ has 3 pivots?
\end{prob}

\begin{prob}
	Consider the matrices
	\[U=\begin{bmatrix}
		1 & 1 & -1\\
		0 & 1 & 2\\
		0 & 0 & 1
	\end{bmatrix}, 
L=\begin{bmatrix}
	1 & 0& 0\\
	-1 & 1 & 0\\
	-2 & 01& 1
\end{bmatrix}, 
B=\begin{bmatrix}
	1 & 2 & 3\\
	3 & 2 & 1\\
	1 & 0 & 1
\end{bmatrix} \]
and set $A=U B^{-1}L$.
Without inverting any matrices, compute the second column of $A^{-1}$.
\end{prob}






\end{document}