\documentclass[11pt]{article}
\usepackage[margin=1in]{geometry}                % See geometry.pdf to learn the layout options. There are lots.
\geometry{letterpaper}                   % ... or a4paper or a5paper or ... 
%\geometry{landscape}                % Activate for for rotated page geometry
%\usepackage[parfill]{parskip}    % Activate to begin paragraphs with an empty line rather than an indent


\renewcommand{\aa}{\mathbb{A}}
\newcommand{\cc}{\mathbb{C}}
\newcommand{\rr}{\mathbb{R}}
\newcommand{\pp}{\mathbb{P}}
\newcommand{\hh}{\mathbb{H}}
\newcommand{\qq}{\mathbb{Q}}
\newcommand{\zz}{\mathbb{Z}}
\newcommand{\ff}{\mathbb{F}}
\newcommand{\kk}{\mathbb{K}}
\renewcommand{\gg}{\mathbb{G}}
\newcommand{\nn}{\mathbb{N}}
\renewcommand{\tt}{\mathbb{T}}

\newcommand{\C}{\mathcal{C}}
\newcommand{\U}{\mathcal{U}}
\newcommand{\I}{\mathcal{I}}
\renewcommand{\H}{\mathcal{H}}
\renewcommand{\O}{\mathcal{O}}
\newcommand{\E}{\mathcal{E}}
\newcommand{\F}{\mathcal{F}}
\newcommand{\G}{\mathcal{G}}
\renewcommand{\P}{\mathcal{P}}
\renewcommand{\S}{\mathcal{S}}
\newcommand{\Q}{\mathcal{Q}}
\newcommand{\T}{\mathcal{T}}
\renewcommand{\L}{\mathcal{L}}
\newcommand{\M}{\mathcal{M}}
\newcommand{\X}{\mathcal{X}}

\newcommand{\sH}{\mathscr{H}}
\newcommand{\sD}{\mathscr{D}}
\newcommand{\sE}{\mathscr{E}}
\newcommand{\sL}{\mathscr{L}}
\newcommand{\sQ}{\mathscr{Q}}
\newcommand{\sX}{\mathscr{X}}




\usepackage{graphicx}
\usepackage{amssymb, amsmath}
\usepackage{epstopdf}
\DeclareGraphicsRule{.tif}{png}{.png}{`convert #1 `dirname #1`/`basename #1 .tif`.png}

\title{18.06 Recitation 10}
\author{Isabel Vogt}
\date{\today}                                           % Activate to display a given date or no date

\begin{document}
\maketitle
%\section{}
%\subsection{}

\begin{enumerate}

\item Consider unitary matrices $Q^HQ = I$.  If $Q$ is real, then we are saying that $Q$ is orthogonal (e.g., $Q^TQ = I$).
\begin{enumerate}
\item Find the flaw in this false proof: \textbf{False Claim: all eigenvalues of a real orthogonal matrix are $\pm 1$.}  Indeed, if $Qx = \lambda x$,
\[\lambda^2 x^T x = (Qx)^T (Qx) = x^T (Q^T Q) x = x^Tx, \]
therefore $\lambda^2 = 1$, so $\lambda = \pm 1$.  If you want, you can think about what happens for the rotation matrices
\[ R = \begin{pmatrix} \cos \theta & -\sin \theta \\ \sin \theta & \cos \theta \end{pmatrix}.\]

\item Correct the proof to show \textbf{True Claim: all eigenvalues of a unitary matrix have magnitude $1$ (e.g. $\lambda = e^{i \phi}$ for some $\phi$).}

\item Show that the eigenvectors for different eigenvalues of a unitary matrix are orthogonal.

\item Show that the determinant of any real unitary matrix (e.g., an orthogonal matrix) is $\pm 1$ using eigenvalues.  (Note: you already proved this on a previous pset in a different way.)

\end{enumerate}

\item Let $A$ be a Hermitian matrix and $B$ be a positive definite Hermitian matrix.  Consider the generalized eigenproblem
\[Ax = \lambda Bx. \]
\begin{enumerate}
\item Show that these ``generalized" eigenvalues $\lambda$ above are real.
\item Show that eigenvectors for different ``generalized eigenvalues" are orthogonal with respect to the ``$B$-dot-product":
\[x_1 \cdot_{B} x_2 = x_1^H B x_2. \]
\end{enumerate}

\item (Strang 6.4, Problem 12) Here is a quick ``proof" that the eigenvalues of \textbf{every} real matrix $A$ are real:
\[\text{\textbf{False Proof: }} Ax = \lambda x \text{ gives } x^TAx = \lambda x^Tx, \quad \text{so } \ \lambda  = \frac{ x^TAx}{x^Tx} = \frac{\text{real}}{\text{real}}.\]
Find the flaw in this reasoning -- a hidden assumption that is not justified.  You can test those steps on the 90${}^\circ$ rotation matrix
\[ A = \begin{pmatrix} 0 & -1 \\  1 & 0 \end{pmatrix}, \qquad \lambda =  i, \ x = \begin{pmatrix} i \\ 1 \end{pmatrix}.\]


\item (Strang 9.2, Problem 23) How are the eigenvalues of $A^H$ related to the eigenvalues of the square matrix $A$?


\item 

\begin{enumerate}
\item If $S$ is a positive definite matrix, show that $S^{-1}$ is also positive definite.
\item If $S$ and $T$ are positive definite, their sum $S + T$ is also positive definite.  If $S = A^HA$ and $T = B^HB$ for full-column-rank matrices $A$ and $B$, then can you write down a full-column-rank matrix $C$ so that $S +T = C^TC$?
\end{enumerate}

\item We saw on homework $2$ that if we represent complex numbers $x + iy$ by the vector of real and imaginary parts $\begin{pmatrix} x \\  y \end{pmatrix}$, then multiplication by the fixed complex number $a + bi$ is given by matrix multiplication by the real matrix
\[\begin{pmatrix} a & -b \\ b & a \end{pmatrix}. \]
What real matrix represents multiplication by $\overline{a + bi}$?











\end{enumerate}
\end{document}  