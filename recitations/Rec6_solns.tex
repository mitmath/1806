\documentclass[11pt]{article}
\usepackage[margin=1in]{geometry}                % See geometry.pdf to learn the layout options. There are lots.
\geometry{letterpaper}                   % ... or a4paper or a5paper or ... 
%\geometry{landscape}                % Activate for for rotated page geometry
%\usepackage[parfill]{parskip}    % Activate to begin paragraphs with an empty line rather than an indent


\renewcommand{\aa}{\mathbb{A}}
\newcommand{\cc}{\mathbb{C}}
\newcommand{\rr}{\mathbb{R}}
\newcommand{\pp}{\mathbb{P}}
\newcommand{\hh}{\mathbb{H}}
\newcommand{\qq}{\mathbb{Q}}
\newcommand{\zz}{\mathbb{Z}}
\newcommand{\ff}{\mathbb{F}}
\newcommand{\kk}{\mathbb{K}}
\renewcommand{\gg}{\mathbb{G}}
\newcommand{\nn}{\mathbb{N}}
\renewcommand{\tt}{\mathbb{T}}

\newcommand{\C}{\mathcal{C}}
\newcommand{\U}{\mathcal{U}}
\newcommand{\I}{\mathcal{I}}
\renewcommand{\H}{\mathcal{H}}
\renewcommand{\O}{\mathcal{O}}
\newcommand{\E}{\mathcal{E}}
\newcommand{\F}{\mathcal{F}}
\newcommand{\G}{\mathcal{G}}
\renewcommand{\P}{\mathcal{P}}
\renewcommand{\S}{\mathcal{S}}
\newcommand{\Q}{\mathcal{Q}}
\newcommand{\T}{\mathcal{T}}
\renewcommand{\L}{\mathcal{L}}
\newcommand{\M}{\mathcal{M}}
\newcommand{\X}{\mathcal{X}}

\newcommand{\sH}{\mathscr{H}}
\newcommand{\sD}{\mathscr{D}}
\newcommand{\sE}{\mathscr{E}}
\newcommand{\sL}{\mathscr{L}}
\newcommand{\sQ}{\mathscr{Q}}
\newcommand{\sX}{\mathscr{X}}




\usepackage{graphicx}
\usepackage{amssymb, amsmath}
\usepackage{epstopdf}
\DeclareGraphicsRule{.tif}{png}{.png}{`convert #1 `dirname #1`/`basename #1 .tif`.png}

\title{18.06 Recitation 6}
\author{Isabel Vogt}
\date{\today}                                           % Activate to display a given date or no date

\begin{document}
\maketitle
%\section{}
%\subsection{}
\section{Problems}

\begin{enumerate}

\item[4.] (essentially Strang, Section 4.4, Problem 33) Find all matrices that are both orthogonal and lower triangular in two parts:
\begin{enumerate}
\item What are all orthogonal and lower-triangular $2\times 2$ matrices?
\item Can you generalize this to $m\times m$ matrices?
\end{enumerate}

\textbf{Solution:} 
\begin{enumerate}

\item Any $2\times 2 $ lower-triangular matrix is of the form
\[\begin{pmatrix} a & 0 \\ c & d \end{pmatrix}. \] 
In order to be orthogonal, we need:
\[a^2 + c^2 = 1, \qquad cd = 0, \qquad d^2 = 1. \]
Let's look at these backwards: first $d^2 = 1$ tells us that $d = \pm 1$.  The second equation tells us that $c = 0$, since we can divide by the nonzero number $d$.  Finally, the first equation simplifies to $a^2 = 1$ and so $a = \pm 1$.
So all orthogonal, lower-triangular $2 \times 2$ matrices are of the form
\[\begin{pmatrix} \pm 1 & 0 \\ 0 & \pm 1 \end{pmatrix}. \]

\item In general, we might surmise that ``working backwards" is helpful.  Let's write our lower-triangular $m \times m $ matrix as
\[Q = \begin{pmatrix} q_1 & \cdots & q_m \end{pmatrix}, \]
where the $i$th component of the $j$th column vector satisfies
\[(q_j)_i = 0 \qquad \text{for } i < j. \]
In particular, for last column $q_m$, the only nonzero entry is in the last position.  So first starting with the condition
\[ q_m \cdot q_m = 1, \]
we see that $((q_m)_m)^2 = 1$ so $(q_m)_m = \pm 1$.
Now for $i < m$, the $i$th column and the $m$th column are orthogonal.  But the only nonzero entry of the $m$th column is the last one, so
\[q_{i} \cdot q_m = (q_i)_m  (q_m)_m = 0 \qquad \Rightarrow \qquad (q_i)_m = 0, \]
since we can divide by the nonzero $(q_m)_m$.  Therefore the bottom row of Q is all zeros except for the final position.

Carrying on, let's look now at the condition $q_{m-1} \cdot q_{m-1} = 1$.  Originally, when we only knew that $Q$ was upper-triangular, $q_{m-1}$ could only have two nonzero entries: $(q_{m-1})_m$ and $(q_{m-1})_{m-1}$.  But from the last step, we learned that $(q_{m-1})_m = 0$.  So $q_{m-1}$ also has only 1 nonzero entry.  Exactly as above we can therefore show that 
\[(q_i)_{m-1} = \begin{cases} 0 & \text{ if $i \neq m-1$} \\ \pm 1 & \text{ if $i = m-1$}. \end{cases} \]
And so we learn that the second to bottom row is also almost all $0$, except in the $(m-1, m-1)$ position.

Carrying on like this, we see that as before, $Q$ must be diagonal 
\[Q =  \begin{pmatrix} \pm 1 &&&& \\ & \pm 1 &&& \\ && \ddots && \\ &&& \pm 1 & \\ &&&& \pm 1 \end{pmatrix}.\]







\end{enumerate}


\end{enumerate}
\end{document}  