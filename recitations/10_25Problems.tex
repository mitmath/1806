\documentclass{article}
\usepackage{amsmath, amsthm, amssymb, amsfonts, dsfont, fancyhdr, graphicx, color, tabularx, enumitem}
\usepackage{geometry}


\theoremstyle{definition}
\newtheorem{prob}{}
\renewcommand{\qedsymbol}{}
\renewcommand*{\proofname}{Solution}
\newcommand{\MSB}[1]{\textcolor{blue}{[MSB: #1]}}

\pagestyle{fancy} \fancyhf{} \lhead{\textsc{18.06}} \rhead{10/25/22} 

\begin{document}


\section*{Practice Problems}
\begin{prob}
	Consider the singular value decomposition $A= U \Sigma V^T$ where
	\[U =\begin{pmatrix}
		u_1 & u_2
	\end{pmatrix}= \begin{pmatrix}
		\frac{1}{\sqrt{3}} & \frac{1}{\sqrt{2}} \\
		\frac{1}{\sqrt{3}}  & 0 \\
		-\frac{1}{\sqrt{3}} & \frac{1}{\sqrt{2}}
	\end{pmatrix} \qquad 
\Sigma= \begin{pmatrix}
	\sigma_1 & 0\\
	0 & \sigma_2
\end{pmatrix}= \begin{pmatrix}
	100 & 0\\
	0 & 1
\end{pmatrix} \qquad
V= \begin{pmatrix}
	v_1 & v_2
\end{pmatrix}= \frac{1}{5} \begin{pmatrix}
3 & 4\\4 &-3
\end{pmatrix}.\]
\begin{itemize}
	\item[a)] What orthonormal bases does this give for $C(A)$ and $C(A^T)$?
	\item[b)] Write $A$ as $\sigma_1 R_1 + \sigma_2 R_2$ where $R_1$ and $R_2$ are rank 1 matrices.
	\item[c)] What is a good rank 1 approximation for $A$?
	\item[d)] If you apply $A$ to a unit circle in $\mathbb{R}^2$, what is the output? (A vague answer is fine.)
	\item[e)] Why not choose $V=I$, which is another orthonormal basis for $C(A^T)$? What does $A$ do to the columns of $I$?
\end{itemize}
\end{prob}

%\begin{proof}
%	a): Given an SVD $A=U \Sigma V^T$, we know that $C(A^T)=C(V)$ and $C(A)= C(U)$. (As a sanity check, $C(A)$ is a subspace of $\mathbb{R}^3$ and $C(A^T)$ is a subspace of $\mathbb{R}^2$, so this is reasonable.) So the orthonormal basis for $C(A)$ is $\{u_1, u_2\}$, and the orthonormal basis for $C(A^T)$ is $\{v_1, v_2\}$.
%	
%	b): We can read $R_1$ and $R_2$ straight off of the formula $A=U \Sigma V^T$.
%	\[A= (u_1 \ u_2) \Sigma \begin{pmatrix}
%		v_1^T \\
%		v_2^T
%	\end{pmatrix}= u_1 \sigma_1 v_1^T + u_2 \sigma_2 v_2^T\]
%so
%\[R_1=u_1 v_1^T= \frac{1}{5\sqrt{3}}\begin{pmatrix}
%	3 & 4\\ 3 & 4\\ -3 & -4
%\end{pmatrix} \qquad \text{ and } \qquad R_2=\frac{1}{5 \sqrt{2}} \begin{pmatrix}
%4 & -3\\ 0 & 0\\4 & -3
%\end{pmatrix}.\]
%
%c): Since $\sigma_1 > \sigma_2$, the better rank 1 approximation of $A$ is $\sigma_1 R_1$.
%
%d): The output is an ellipse in $\mathbb{R}^3$. 
%
%e): The point of choosing $V$ as in the problem is that $A$ takes the columns of $V$ to another set of orthogonal vectors (more precisely, $Av_i= \sigma_i u_i$, so the $\sigma_i$'s tell us how long the output vectors are). If we chose $V=I$, then the output vectors $Ae_1, Ae_2$ are just the columns of the matrix $A$ (which you can compute by multiplying out $U \Sigma V^T$). Because $\sigma_1$ is so much larger than $\sigma_2$, we can see that the first and second columns of $A$ are quite close together, so are far from orthogonal.
%\end{proof}

\begin{prob}
	Suppose $A$ is square and upper triangular (with nonzero diagonal entries). If you perform Gram-Schmidt on the columns of $A$, what can you say about the square matrix $Q$ whose columns are the Gram-Schmidt vectors?
\end{prob}

%\begin{proof}
%	Performing Gram-Schmidt is the same as factoring $A$ into $A=QR$ where $Q$ has orthonormal columns and $R$ is a square upper-triangular matrix. If $A$ is already upper-triangular, then $Q$ will be very close to the identity matrix; it will be a diagonal matrix with $\pm1$'s on the diagonal. If this is confusing, try running Gram-Schmidt on a small square upper-triangular matrix, like 
%	\[\begin{pmatrix}
%		-2& 3\\
%		0 &4
%	\end{pmatrix}\]
%\end{proof}

\begin{prob}
	\begin{itemize}
		\item[a)] Give a $4 \times 3$ matrix $A$ with 3 different, nonzero columns such that blindly applying Gram-Schmidt to the columns of $A$ will lead you to divide by zero.
		\item[b)] What property of $A$ causes Gram-Schmidt to fail?
		\item[c)] To find an orthonormal basis for $C(A)$, you should instead apply Gram-Schmidt to what matrix $B$?
	\end{itemize}
\end{prob}

%\begin{proof}
%	For a): Say the columns of $A$ are $a_1, \dots, a_n$. In the Gram-Schmidt algorithm, you take $a_i$ and subtract the projection of $a_i$ onto the span of $a_1, \dots, a_{i-1}$ to get some vector $v_i$. Then you divide by the norm of $v_i$ in order to get a vector of norm 1. We want to construct $A$ so that at some point in this process, we divide by zero. In the algorithm, we only ever divide by the norm of the vector $v_i$. So we'd like to choose $A$ so that some $v_i$ is the zero vector; that is, some column $a_i$ is in the span of the previous columns. One example is 
%	\[A=\begin{pmatrix}
%		1& 2& 5\\
%		1& 2& 5\\
%		1& 1 & 3
%	\end{pmatrix}.\]
%For this $A$, $a_3=a_1 + 2a_2$. There are many other examples.
%
%For b): The reasoning in the above paragraph tells us that the reason Gram-Schmidt failed (i.e. the reason we divided by zero) is that $A$ does not have full column rank.
%
%For c): We just need to throw out enough columns that we're left with a basis for the original column space. Applying Gram-Schmidt to the resulting matrix $B$ won't run into any problems. For the example above, 
%\[B=\begin{pmatrix}
%	1& 2\\
%	1& 2\\
%	1& 1
%\end{pmatrix}\]
%would work (so would the matrix consisting of any two columns of $A$).
%\end{proof}


\end{document}