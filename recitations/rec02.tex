\documentclass{article}
\usepackage{amsmath, amsthm, amssymb, amsfonts, dsfont, fancyhdr, graphicx, color, tabularx, enumitem}
\usepackage{geometry}


\theoremstyle{definition}
\newtheorem{prob}{}
\renewcommand{\qedsymbol}{}
\renewcommand*{\proofname}{Solution}
\newcommand{\MSB}[1]{\textcolor{blue}{[MSB: #1]}}

\pagestyle{fancy} \fancyhf{} \lhead{\textsc{18.06}} \rhead{9/20/22} 

\begin{document}


\section*{Practice Problems}
\begin{prob}
	Say $P$ is a permutation. Why is $P^{-1}=P^T$? (You're allowed to use this fact in the pset without justification, but it's good to know why it's true! If you get stuck, answer this question for a particular permutation.)
	
	\vspace{12pt}
	
	\noindent\textbf{Bonus question:} Say $P[x_1, \dots, x_n]=[x_{p_1}, \dots, x_{p_n}]$ and $P^{-1}[x_1, \dots, x_n]=[x_{q_1}, \dots, x_{q_n}]$. What's the relationship between the lists $(p_1, \dots, p_n)$ and $(q_1, \dots, q_n)$?
	
\end{prob}

\begin{prob}
	Say 
	\[A=\begin{bmatrix}
		1& 3& 0& 1\\
		0&1&0&2\\
		0&0&1&1\\
		0&0&0&1
	\end{bmatrix}
\begin{bmatrix}
	-1& 0& 0& 0\\
	0&2&0&0\\
	0&0&-3&0\\
	0&0&0&-2
\end{bmatrix}
\begin{bmatrix}
	0& 0& 0& 1\\
	0&0&1&-1\\
	0&1&0&3\\
	1&1&0&1
\end{bmatrix}.\]
Solve 
\[Ax=\begin{bmatrix}
	12\\6\\32\\2
\end{bmatrix}\]
for $x$ \emph{without} using Gaussian elimination.
\end{prob}

\begin{prob}
	Consider the following \emph{tridiagonal} matrix
	\[A=\begin{bmatrix}
		1 & -1& 0&0&0\\
		2&-1&1&0&0\\
		0&3&4&-2&0\\
		0&0&-2&5&-2\\
		0&0&0&-1&3\\
	\end{bmatrix}.\]
\begin{itemize}
	\item[a)] Compute its $LU$ factorization using Gaussian elimination. What do you notice about the pattern of nonzero entries?
	\item[b)] Compute the 3rd column of $A^{-1}$. What do you notice about the pattern of nonzero entries?
	\item[c)] Suppose you carried out arithmetic at the same rate, but $A$ was a $250 \times 250$ tridiagonal matrix instead of $5 \times 5$. How much longer would a) have taken?
\end{itemize}
\end{prob}

\begin{prob} Are the following subsets of $\mathbb{R}^2$ vector spaces? Why or why not?
	\begin{itemize}
		\item [a)] $[x, y]$ satisfying $mx+b=y$, where $m, b$ are fixed scalars.
		\item [b)] $[x,y]$ satisfying $x^2 + y^2 =1$ (the unit circle).
		\item [c)] $[x, y]$ satisfying 
		\[A \begin{bmatrix}
			x \\y
		\end{bmatrix}=0\]
	where $A$ is a fixed $2 \times 2$ matrix.
	\end{itemize}
\end{prob}


\end{document}