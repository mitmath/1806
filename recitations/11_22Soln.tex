\documentclass{article}
\usepackage{amsmath, amsthm, amssymb, amsfonts, dsfont, fancyhdr, graphicx, color, tabularx, enumitem}
\usepackage{geometry}


\theoremstyle{definition}
\newtheorem{prob}{}
\renewcommand{\qedsymbol}{}
\renewcommand*{\proofname}{Solution}
\newcommand{\MSB}[1]{\textcolor{blue}{[MSB: #1]}}
\def\tr{\text{tr}}

\pagestyle{fancy} \fancyhf{} \lhead{\textsc{18.06}} \rhead{11/22/22} 

\begin{document}


\section*{Practice Problems}
\begin{prob}
	Say $A$ is a $3 \times 3$ real matrix with eigenvalues $\lambda_1=-1, \lambda_2=-3+4i, \lambda_3=-3-4i$, with corresponding eigenvectors $x_1, x_2, x_3$. 
	\begin{itemize}
		\item[a)] What are the trace and determinant of $2A$?
		\item[b)] Two eigenvectors of $A$ are $x_1=\begin{pmatrix}
			1\\0\\0
		\end{pmatrix}$
	and 
	$x_2=\begin{pmatrix}
		0\\1\\i
	\end{pmatrix}$.
What is $x_3$?
	\end{itemize}
\end{prob}

\begin{proof}
	\begin{itemize}
		\item[a)] The eigenvalues of $2A$ are $2\lambda_1, 2\lambda_2, 2\lambda_3$. So the trace is
	\[2\lambda_1+ 2\lambda_2+ 2\lambda_3= 2(-1+ (-3+4i)+ (-3-4i))= 2(-7)=-14 \]
	and the determinant is 
	\[2\lambda_1\cdot 2\lambda_2\cdot 2\lambda_3= 8 (-1) (-3+4i)(-3-4i)= -200.\]
	\item[b)] Since $A$ is real, its complex eignevectors must come in complex-conjugate pairs. So $$x_3= \overline{x}_2=\begin{pmatrix}
		0 \\1\\-i
	\end{pmatrix}.$$
	\end{itemize}
\end{proof}

\begin{prob}
	Using the same matrix $A$ as in 1, which of the following has unbounded magnitude (i.e. magnitude blowing up) as $n \to \infty$ or $t \to \infty$? Assume $y$ is chosen at random.
	\begin{itemize}
		\item[a)] $A^n y$ as $n \to \infty$
		\item[b)] $A^{-n}y$ as $n \to \infty$
		\item[c)] The solution of $\frac{dx}{dt}=Ax$ as $t \to \infty$ for the initial condition $x(0)=y$.
		\item[d)] The solution of $\frac{dx}{dt}=-Ax$ as $t \to \infty$ for the initial condition $x(0)=y$.
	\end{itemize}
\end{prob}

\begin{proof}
	Notice that the eigenvalues of $A$ satisfy $|\lambda|\geq 1$ and $\mathrm{Re}[\lambda]<0$. This is enough information for us to answer the first 4 parts of this question.
	\begin{itemize}
		\item[a)] This has unbounded magnitude. If we write $y= c_1 x_1 + c_2 x_2 + c_3 x_3$, then 
		\[A^n y=c_1 \lambda_1^n x_1 + c_2 \lambda_2^n x_2 + \lambda_3^n c_3 x_3 \]
		and $\lambda_2^n, \lambda_3^n$ become larger and larger in magnitude as $n \to \infty$ (you can see this by writing those eigenvalues in polar form). Since $y$ was chosen at random, $c_2, c_3$ are likely nonzero.
		\item[b)] The magnitude of this vector will stay bounded as $n \to \infty$ (though it may not converge to any vector in particular). Remember the eigenvalues of $A^{-1}$ are $1/\lambda_i$ and $|1/\lambda_i|\leq 1$. So writing
		\[A^{-n} y=c_1 \lambda_1^{-n} x_1 + c_2 \lambda_2^{-n} x_2 + \lambda_3^{-n }c_3 x_3 \]
		we see that the second and last term will decay as $n \to \infty$ (e.g. by writing those eigenvalues in polar form). The first term will always have the same magnitude.
		\item[c)] The solution to this equation is 
		\[x(t)= e^{At}y=c_1 e^{\lambda_1t}x_1 + c_2 e^{\lambda_2t}x_2 +c_3 e^{\lambda_3t}x_3\]
		and it has bounded magnitude as $t \to \infty$. This is because $\mathrm{Re}[\lambda_j]<0$ for all eigenvalues, so $e^{\lambda_jt}$ always approaches zero as $t \to \infty$ (you can see this by writing $\lambda_j= a+ ib$). 
		\item[d)] The eigenvalues of $-A$ are $-\lambda_j$, so they all have positive real parts. This means that the solution 
			\[x(t)= e^{-At}y=c_1 e^{-\lambda_1t}x_1 + c_2 e^{-\lambda_2t}x_2 +c_3 e^{-\lambda_3t}x_3\]
			will have unbounded magnitude as $t \to \infty$, since each term has magnitude which blows up.
	\end{itemize}
\end{proof}


\begin{prob}
	Using the same matrix $A$ as in 1, write down the exact solution $x(t)$ to $\frac{dx}{dt}= Ax$ for the initial condition $x(0)= \begin{pmatrix}
		1\\2\\0
	\end{pmatrix}$.
\end{prob}

\begin{proof}
	As in part c) above, the general solution to $\frac{dx}{dt}= Ax$ is 
			\[x(t)= e^{At}x(0)=c_1 e^{-1t}x_1 + c_2 e^{(-3+4i)t}x_2 +c_3 e^{(-3-4i)t}x_3\]
			where $c_1, c_2, c_3$ are some constants depending on $x(0)$. Because the initial conditions are real, we expect $c_2=\overline{c_3}$. 
			
			Setting $t=0$, we get
			\[x(0)= c_1 x_1 + c_2 x_2+c_3 x_3= \begin{pmatrix}
				1\\2\\0
			\end{pmatrix}.\]
		
		Eyeballing, we see that this is true if $c_1=c_2=c_3=1$, so the exact solution is 
		\[x(t)=  e^{-1t}x_1 + e^{(-3+4i)t}x_2 + e^{(-3-4i)t}x_3.\]
\end{proof}

\begin{prob}
	Use the series formula for $e^{At}$ to show that
	\[\frac{d}{dt}e^{At}= A e^{At}.\]
	Use this to conclude that $x(t)= e^{At}x(0)$ satisfies $\frac{dx}{dt}= Ax$.
\end{prob}

\begin{proof}
	The series formula for $e^{At}$ is 
	\[e^{At}= 1+ At + \frac{A^2t^2}{2!}+ \frac{A^3t^3}{3!}+\dots \]
	(Note: $t$ is a scalar here! So $A^nt^n$ is well-defined). We want to take the derivative with respect to $t$; that is, we treat $A$ as a constant and $t$ as a variable. Taking derivatives ``term by term" in the series gives 
	\begin{align*}
		\frac{d}{dt} e^{At}&= A + 2 \frac{A^2 t}{2!} + 3 \frac{A^3t^2}{3!}+ \dots \\
		&= A(1+ At + \frac{A^2t^2}{2!}+ \frac{A^3t^3}{3!}+\dots )\\
		&= A e^{At}.
	\end{align*}

Now we compute 
\begin{align*}
	\frac{d}{dt}(e^{At}x(0))&= x(0)\left(\frac{d}{dt}(e^{At}) \right)\\
	&= x(0)A e^{At}\\
	&= A (e^{At}x(0))
\end{align*}
which shows the property we were supposed to show.
	
\end{proof}
\end{document}